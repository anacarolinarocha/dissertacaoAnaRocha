\label{cap:Resultados}%
Neste capítulo apresenta-se o resultado do estudo e experimentos iniciais que foram obtidos até o momento.

\section{\label{sec:entendimentoNegocio}Entendimento do Negócio}

Nesta etapa, busca-se entender a relevância do assunto dentro do contexto jurídico trabalhista, bem como se dá o fluxo processual dentro do PJe. Parte deste entendimento já se encontra apresentado nas Seções \ref{sec:justicaDoTrabalho} e \ref{sec:pje}. Como o escopo do projeto está restrito aos processos que tramitam na segunda instância, é preciso entender quais são os documentos juntados no momento em que o processo chega nesta jurisdição. 

Para que um processo seja autuado neste nível, existem duas possibilidades. A primeira delas trata dos processos que são originários na segunda instância, ou seja, a competência do julgamento da lide em questão deve ser apreciada pela segunda instância, não cabendo julgamento pela primeira instância. Assim, os advogados protocolam o documento do tipo Petição Inicial diretamente na segunda instância. A segunda possibilidade trata dos processos que já foram julgados em primeira instância, mas que uma das partes (ou ambas) discordam da decisão dos magistrados, de forma que recorrem à segunda instância para que julgue os itens em discordância. Os advogados podem recorrer elaborando um dos seguintes tipos documentos, que são protocolados ainda em primeira instância: Agravo de Instrumento em Agravo de Petição, Agravo de Instrumento em Recurso Ordinário, Agravo de Petição, Recurso Adesivo, Recurso Ordinário. Uma vez anexados esses documentos ao processo, e preenchida as demais informações necessárias, o processo é remetido da primeira à segunda instância, e passa a tramitar neste grau. Os demais documentos que são juntados posteriormente à chegada do processos são menos relevantes para a análise pois nesse momento, uma vez que os servidores já terão que ter lido os documentos iniciais, e já terão tido o esforço manual de identificar manualmente o assunto do processo.  Abaixo explica-se resumidamente o propósito de cada documento considerando o conjunto de documentos que chegam à 2ª instância:


\begin{itemize}
\item \textbf{Petição inicial}: É a peça processual que dá início ao processo, abrangendo os fatos ocorridos, os fundamentos jurídicos e o pedido, apresentando ao juiz as informações necessárias para o julgamento da causa. Está descrita no artigo 319 do CPC \cite{novocpc}.

\item \textbf{Recurso Ordinário}:Trata-se de recurso de fundamentação livre cabível contra sentenças definitivas e terminativas proclamadas na primeira instância buscando uma reforma da decisão judicial que foi elaborada por um órgão hierarquicamente superior. O Recurso Ordinário é regulamentado pelo artigo 895 da Consolidação das Leis do Trabalho (CLT)\cite{recursos}.

\item \textbf{Agravo de petição}: Recurso utilizado para contestar decisões definitivas que aconteceram na fase execução, visando rediscutir a penhora e os cálculos da liquidação \cite{recursos,agravopeticao}.

\item \textbf{Recurso adesivo}: Acontece quando as duas partes envolvidas no processo vencem e são vencidas em um processo. Assim, se uma das partes discordar da decisão e entrar com um recurso, a outra parte, ainda que inicialmente tenha optado por não recorrer à segunda instância dentro do prazo inicial, poderá protocolar recurso adesivo fora do prazo recursal original. É um recurso subordinário, uma vez que está condicionado à existência do recurso principal protocolado pela parte contrária \cite{recursos}

\item \textbf{Agravo de instrumento}: é uma forma de contestar decisão que tenha negado que um recurso já protocolado subisse à instância superior, ou seja, a parte entrou com um recurso, o recurso foi negado dentro da mesma instância, então a parte entra com agravo de instrumento para que ainda assim o processo seja apreciado pela instância superior. O agravo de instrumento pode ser sobre o Agravo de petição ou sobre o Recurso Ordinário, o que caracteriza os documentos de Agravo de Instrumento em Agravo de Petição e o Agravo de Instrumento em Recurso Ordinário, respectivamente. \cite{agravoinstrumento}
\end{itemize}

    Assim, é natural que nestes documentos estejam contidas as informações relacionadas ao assunto que devem ser julgados pela 2ª instância, sendo portanto,  documentos apropriados para a identificação dos assuntos processuais. Conforme se pode ler no Manual de Utilização das Tabelas Processuais Unificadas do Poder Judiciário \cite{manualtpucnj}, pode-se identificar o assunto do processo no documento de petição inicial, na parte que se refere aos fatos, logo após a citação das partes, ou na parte final. Já em graus recursais, como por exemplo o 2º grau, o assunto pode ser identificado nos documentos que contém o relatório da decisão recorrida. \label{sec:entendimentoNegocio}.
    

%%%%%%%%%%%%%%%%%%%%%%%%%%%%%%%%%%%%%%%%%%%%%%%%%%%%%%%%%%%%%%%%%%%%%%%%%%%%%%%%
%%%%%%%%%%%%%%%%%%%%%%%%%%%%%%%%%%%%%%%%%%%%%%%%%%%%%%%%%%%%%%%%%%%%%%%%%%%%%%%%
%%%%%%%%%%%%%%%%%%%%%%%%%%%%%%%%%%%%%%%%%%%%%%%%%%%%%%%%%%%%%%%%%%%%%%%%%%%%%%%%

\section{Entendimento dos dados}%

Nesta etapa, busca-se entender os dados que serão trabalhados por meio de uma análise exploratória dos dados. 

\subsection{Assuntos}



O objeto de estudo deste trabalho está relacionado com a identificação do assunto processual. A TPU de Assuntos Processuais define a terminologia jurídica a ser utilizada para a categorização dos processos, e esta informação se refere ao conteúdo, à temática, à matéria do processo. Baseado na tabela disponibilizada pelo CNJ \cite{tpucnj}, o TST, dentro de sua competência, acrescentou assuntos específicos da JT que entendeu serem necessários à este ramo da justiça e removeu aqueles que não cabem neste contexto, transformando a TPU original em uma TPU específica para a JT \cite{tputst}.

A tabela de assuntos está organizada de forma hierárquica, e quanto maior o nível do assunto, mais especificado estará. Atualmente, a tabela conta com 5 níveis de hierarquia, sendo 1131 assuntos ao todo, que estão distribuídos conforme \refTab{quantidadesAssuntos}. Na \refFig{extratoTabelaAssuntos}, tem-se um exemplo extraído da TPU de Assuntos Processuais do TST \cite{tputst}, onde pode-se entender como está organizada esta informação: no primeiro nível tem-se o assunto ``Direito do Trabalho'', abaixo dele tem-se o segundo nível com o assunto ``Categoria Profissional Especial'', em seguida, ocupando um 3º nível, tem-se o assunto ``Bancário'', do qual mostra-se 5 assuntos filhos no nível 4, sendo um deles o ``Enquadramento'', dividido em 4 especializações de 5º nível. À direita dos assuntos, pode-se encontrar o código único de cada um.

\figuraBib{ExtratoTabelaAssuntos}{Extrato da Tabela de Assuntos}{tputst}{extratoTabelaAssuntos}{width=1\textwidth}%

\begin{table}[]
\centering
\caption{Quantidades de Assuntos}
\label{quantidadesAssuntos}
\begin{tabular}{|c|c|}
\hline
Nível de Hierarquia & Quantidade de Assuntos \\ \hline
1                   & 5                      \\ \hline
2                   & 57                     \\ \hline
3                   & 420                    \\ \hline
4                   & 539                    \\ \hline
5                   & 110                    \\ \hline
Total               & 1131                   \\ \hline
\end{tabular}
\end{table}

Fazendo então uma análise da distribuição dos processos de acordo com os assuntos, considerando consulta executada nas bases de 2º grau de 20\footnote{O CSJT tem acesso à uma cópia da base dos 25 Tribunais, entretanto, nem todas estão disponíveis 24 horas por dia, podendo haver eventuais indisponibilidades.} TRTs, considerando o nível do assunto principal em relação à quantidade de processos (eixo y), tem-se a imagem da \refFig{qtdProcessosNivelAssunto}, onde nota-se que a maior parte dos processos está categorizada com assuntos do nível 4. Importante mencionar que nem todos os assuntos possuem um nível mais detalhado (como 4 ou 5 por exemplo). Assim, de forma a delimitar o escopo do projeto escolhe-se abordar neste trabalho uma classificação cujo escopo seja avaliar os processos de acordo com o nível 3 da tabela.

\figuraBib{DitribuicaoProcessosPorNivelAssunto}{Distribuição da quantidade de processos de acordo com o nível do assunto principal}{}{qtdProcessosNivelAssunto}{width=1\textwidth}%

\subsection{Documentos}

A partir de uma consulta nas bases de dados de , fez-se uma análise da quantidade de documentos relevantes para o contexto da pesquisa.

    
     A \refFig{totalGeralDocumentos} mostra como estes documentos se distribuem. Nota-se que a maior parte de documentos são do tipo Recurso Ordinário.
%Por apresentar maior quantidade, este é o tipo de documento escolhido para realizar os experimentos iniciais nesta primeira iteração do projeto.


\figuraBib{totalGeralDocumentos}{Quantitativo de documentos na segunda instância}{}{totalGeralDocumentos}{width=1\textwidth}%



%A \refFig{totalGeralDocumentos}  mostrou um panorama geral da distribuição dos documentos, considerando todos os Tribunais Regionais disponíveis para consulta. Entretanto, há que se considerar a forma como o sistema PJe está implantado na Justiça do Trabalho. Uma vez que as bases de dados de cada Tribunal Regional são separadas umas das outras e não há qualquer tipo de integração, a análise dos documentos deve ser restrita à cada Região, onde cada uma fará a análise dos documentos que possui. Neste caso, para os experimentos a serem realizados neste projeto, escolhe-se a base de dados do Tribunal Regional do Trabalho da 3ª Região, que representa o estado de Minas Gerais. Escolhe-se este Tribunal devido à proximidade com dois magistrados deste órgão, que atualmente compõe o Grupo de Negócios da CNE, e poderão auxiliar no processo de desenvolvimento, atuando como especialistas. 

%%%%%%%%%%%%%%%%%%%%%%%%%%%%%%%%%%%%%%%%%%%%%%%%%%%%%%%%%%%%%%%%%%%%%%%%
%%%%%%%%%%%%%%%%%%%%%%%%%%%%%%%%%%%%%%%%%%%%%%%%%%%%%%%%%%%%%%%%%%%%%%%%
%%%%%%%%%%%%%%%%%%%%%%%%%%%%%%%%%%%%%%%%%%%%%%%%%%%%%%%%%%%%%%%%%%%%%%%%

\section{Preparação dos dados}%

De forma a se ter acesso ao conteúdo dos documentos, é preciso acessar o conteúdo de cada arquivo, que se encontra armazenado atualmente no banco Postgres. Parte destes documentos se encontram armazenados em texto puro, no formato HTML, enquanto outros se encontram armazenados em PDF. Atualmente, há um projeto em desenvolvimento que faz a inserção destes dados na ferramenta Solr, que é específica para o armazenamento de textos e recuperação da informação. Um dos artefatos do projeto é um extrator, que lê os dados no Postgres e insere no Solr. O extrator faz uso do Apache Tika para a extração do conteúdo dos documentos PDF não escaneados, e usa a ferramenta Tesseract para fazer o reconhecimento óptico dos caracteres (OCR) dos documentos escaneados. 


%Enquanto o projeto não está concluído, fez-se uso do extrator para a extração dos documentos armazenados em HTML. Eles foram inseridos em uma instância do Solr local, de forma que se pôde acessar os documentos

%De forma a se ter um primeiro contato com os dados, enquanto o projeto que fará a indexação dos documentos não está concluído, acessou-se diretamente a base do Postgres com a ferramenta Pentaho, e gerou-se um arquivo csv com o conteúdo de um subconjunto dos Recursos Ordinários gerados na base de 1º grau do TRT 3. 

\subsection{Pré-processamento}%


Para que se tenha um bom resultado nos modelos preditivos, inicialmente faz-se necessária a aplicação de técnicas de pré-processamento de texto, que envolvem a remoção \textit{stopwords}, a stemização, e a transformação de todo o texto em caixa baixa (ou caixa alta), para que não haja diferenças entre palavras iguais utilizando caracteres maiúsculos ou minúsculos. A partir do uso das bibliotecas scikit-learn, BeautifulSoup e nltk da linguagem Python, é possível realizar este trabalho inicial. Será avaliado o desempenho dos algoritmos de classificação escolhidos com uma representação dos textos gerado pela técnica td-idf e outra por uma pela combinação dos\textit{ word embeddings} de cada palavra que compõe os documentos.

\subsection{\label{sec:reducaoDimensionalidade}Redução da dimensionalidade}%

Como se estará trabalhando com textos jurídicos, e que apresentam uma grande quantidade de palavras, possivelmente será preciso adotar técnicas de redução da dimensionalidade de forma a reduzir tempo e custo na aplicação dos algoritmos. Conforme revisão do estado da arte, pretende-se aplicar os algoritmos LDA, SVD e PCA para a redução da dimensionalidade. 

Uma vez aplicados estes métodos, serão aplicados os algoritmos de classificação, e aquele que apresentar melhor precisão com as diferentes possibilidades será escolhido como forma de redução para o artefato de software que será gerado.

%\subsection{Remoção de ruído}%

%Além do pré processamento do texto em si, uma vez que se sabe que há um elevado índice de assuntos classificados erradamente, é possível que seja necessário tentar buscar os registros que foram classificados corretamente por meio da aplicação de técnicas de redução de ruído, tal como foi feito em \cite{tripadivisor}, onde se encontra o centroide de cada classe e busca-se apenas os elementos mais próximos aos centroides no conjunto de treinamento. Neste trabalho \cite{tripadivisor} , utilizou-se apenas o TF-IDF como métrica para avaliar a distância entre os elementos, mas pode-se escolher os vetores gerados pelo método de redução de dimensionalidade escolhido para que se tenha um vetor de menor tamanho para esta análise.


%%%%%%%%%%%%%%%%%%%%%%%%%%%%%%%%%%%%%%%%%%%%%%%%%%%%%%%%%%%%%%%%%%%%%%%%
%%%%%%%%%%%%%%%%%%%%%%%%%%%%%%%%%%%%%%%%%%%%%%%%%%%%%%%%%%%%%%%%%%%%%%%%
%%%%%%%%%%%%%%%%%%%%%%%%%%%%%%%%%%%%%%%%%%%%%%%%%%%%%%%%%%%%%%%%%%%%%%%%

\section{\label{sec:criacaoModelos}Criação de modelos}%

A partir da experiência demonstrada nos trabalhos relacionados listados no Capítulo \ref{cap:TrabalhosRelacionados}, quer-se aplicar os algoritmos que tem apresentado o melhor resultado na literatura. Dentre os algoritmos utilizados para a classificação de textos levantados, escolheu-se aplicar o SVM,\textit{ Random Forest}, Naïve Bayes e as Redes Neurais. 

Além dos modelos que serão criados baseados apenas na análise dos textos, quer-se ainda extrair metadados do processo como por exemplo classe judicial, jurisdição, data do protocolo, de forma a verificar se os modelos apresentarão melhores resultados com o acréscimo destas informações.

Considerando que este trabalho é a primeira iniciativa de aprendizagem de máquina, torna-se relevante que haja confiabilidade dos resultados apresentados, assim, a métrica precisão será priorizada. Serão avaliadas ainda a revocação e \textit{F-Measure}.



%%%%%%%%%%%%%%%%%%%%%%%%%%%%%%%%%%%%%%%%%%%%%%%%%%%%%%%%%%%%%%%%%%%%%%%%
%%%%%%%%%%%%%%%%%%%%%%%%%%%%%%%%%%%%%%%%%%%%%%%%%%%%%%%%%%%%%%%%%%%%%%%%
%%%%%%%%%%%%%%%%%%%%%%%%%%%%%%%%%%%%%%%%%%%%%%%%%%%%%%%%%%%%%%%%%%%%%%%%

\section{Avaliação}%
O escopo deste trabalho está delimitado até a avaliação matemática do modelo, por meio das métricas descritas na Seção \ref{sec:criacaoModelos}. Uma vez atingidos valores aceitáveis para a precisão, onde concluí-se o trabalho, os resultados serão encaminhados para uma equipe especialista, de forma que possam avaliar o resultado pela ótica do negócio.


%%%%%%%%%%%%%%%%%%%%%%%%%%%%%%%%%%%%%%%%%%%%%%%%%%%%%%%%%%%%%%%%%%%%%%%%
%%%%%%%%%%%%%%%%%%%%%%%%%%%%%%%%%%%%%%%%%%%%%%%%%%%%%%%%%%%%%%%%%%%%%%%%
%%%%%%%%%%%%%%%%%%%%%%%%%%%%%%%%%%%%%%%%%%%%%%%%%%%%%%%%%%%%%%%%%%%%%%%%
\section{Implantação} 

Uma vez que este estudo se limita à avaliação matemática dos modelos, não está prevista a implantação. Será gerado um artefato de software final, cujo código gerado será estruturado de forma que, para um determinado conjunto de dados, haja o particionamento das informações, de forma a se separar o conjunto de treinamento e validação, e que sejam testados os 4 algoritmos distintos com o método de redução de dimensionalidade escolhido, conforme descrito na Seção \ref{sec:reducaoDimensionalidade}. A métrica de precisão será utilizada para escolher o algoritmo de melhor acerto, e a métrica \textit{F-measure} será utilizada para desempate, se houver.


%%%%%%%%%%%%%%%%%%%%%%%%%%%%%%%%%%%%%%%%%%%%%%%%%%%%%%%%%%%%%%%%%%%%%%%%
%%%%%%%%%%%%%%%%%%%%%%%%%%%%%%%%%%%%%%%%%%%%%%%%%%%%%%%%%%%%%%%%%%%%%%%%
%%%%%%%%%%%%%%%%%%%%%%%%%%%%%%%%%%%%%%%%%%%%%%%%%%%%%%%%%%%%%%%%%%%%%%%%

\section{Ferramentas e Software de Apoio}%

Para este trabalho, será utilizado o Pentaho para extração e cruzamento de dados, o Apache Tika e o Tesseract para a recuperação do conteúdo textual de documentos escaneados, o Solr para a indexação dos textos, a linguagem Python para o pré-processamento do texto e o H20 para a criação da maior parte dos modelos. 


